%%%%%%%%%%%%%%%%%%%%%%%%%%%%%%%%%%%%%%%%%%%%%%%%%%%%%%%%%%%%%%%%%%%%%%%%%%%%%%%
\section{Test Case 2}
\label{sec:test-case2}
%%%%%%%%%%%%%%%%%%%%%%%%%%%%%%%%%%%%%%%%%%%%%%%%%%%%%%%%%%%%%%%%%%%%%%%%%%%%%%%

The second test case was designed to investigate the impact of the groupwise reaction rate errors in the context of an eigenvalue calculation for a PWR benchmark. The influence of the energy-integrated reaction rate errors on the global eigenvalue is evaluated in~\autoref{sec:case2-eigenvalue-bias}, while ~\autoref{sec:case2-flux-bias} quantifies the relationship between the bias and reaction rate errors in the energy groups with large resonant capture cross sections.

%%%%%%%%%%%%%%%%%%%%%%%%%%%%%%%%%%%%%%%%%%%%%%%%%%%%%%%%%%%%%%%%%%%%%%%%%%%%%%%
\subsection{Eigenvalue Bias}
\label{sec:case2-eigenvalue-bias}

The OpenMC Monte Carlo code was used to tally MGXS in seventy energy groups and to compute a reference eigenvalue for the second test case benchmark. Reference eigenvalues are presented in~\autoref{tab:keff-reference} for simulations with ``normal'' anisotropic scattering, and for the case when OpenMC's ``iso-in-lab'' feature was employed. The two reference OpenMC simulations were also used to tally separate MGXS libraries to quantify the isotropic in lab scattering approximation used in OpenMOC and to isolate it from the flux separabilty approximation.

\begin{table}[h!]
  \centering
  \caption{Reference OpenMC eigenvalues for a 2D fuel pin.}
  \label{tab:keff-reference} 
  \begin{tabular}{c c}
  \toprule
  {\bf Anisotropic} &
  {\bf Isotropic in Lab} \\
  \midrule
  1.17488 $\pm$ 0.00001 & 1.17422 $\pm$ 0.00001 \\
  \bottomrule
\end{tabular}
\end{table}

The MGXS were employed by a series deterministic multi-group transport simulations to quantify the interaction between the energy and spatial approximations. The effects of energy discretization was analyzed by collapsing the 70-group MGXS library to coarser group structures used by the CASMO code. The MOC Flat Source Region (FSR) spatial discretization meshes were varied with constant-by material MGXS in each FSR to quantify the interaction between the energy and spatial approximations. In particular, 1, 4 or 16 radial rings were used to discretize both the fuel and moderator, while 8 azimuthal sectors were used to discretize the fuel, gap, clad and moderator. The MGXS were computed using tally meshes in OpenMC identical to the FSR meshes used by OpenMOC. All OpenMOC simulations used 128 azimuthal angles and 0.01 cm track spacing for the characteristic track laydown.

%The results underline the complex interactions between discretizations in energy and space which are impacted by the loss of angular information due ot the flux separability approximation. 

In the results that follow, the bias $\Delta\rho$ compares the eigenvalue $k_{eff}^{MOC}$ computed by OpenMOC to that of the reference eigenvalue $k_{eff}^{MC}$ computed by OpenMC in units of pcm:

\begin{equation}
\label{eqn:delta-rho}
\Delta\rho = \left(k_{eff}^{MOC} - k_{eff}^{MC}\right) \times 10^{5}
\end{equation}

\noindent The eigenvalue bias for the ``normal'' anisotropic and ``iso-in-lab'' simulations are presented in~\autoref{tab:keff-bias-aniso} and~\autoref{tab:keff-bias-iso-in-lab}, respectively, for varying energy group structures and FSR spatial discretizations. The results illustrate a strong interaction between the energy and spatial meshes used to solve the multi-group transport equation.

%In particular, the eigenvalue bias grows in magnitude with more energy groups and FSRs but is largely insensitive to the the elimination of the isotropic scattering approximation.

\begin{table}[h!]
  \centering
  \caption{The eigenvalue bias with anisotropic scattering.}
  \label{tab:keff-bias-aniso} 
  \begin{tabular}{c S[table-format=6.1] S[table-format=6.1] S[table-format=6.1]}
  \toprule
  & \multicolumn{3}{c}{{\bf FSR Discretization}} \\
  \midrule
  \multicolumn{1}{c}{{\bf \# Groups}} &
  {\bf 1$\times$} & {\bf 4$\times$} & {\bf 16$\times$} \\
  \midrule
1 & 67 & 63 & 92 \\
2 & 22 & -56 & -51 \\
4 & -58 & -128 & -135 \\
8 & -75 & -182 & -197 \\
16 & -73 & -190 & -207 \\
25 & -128 & -246 & -268 \\
40 & -131 & -261 & -288 \\
70 & -132 & -267 & -297 \\
  \bottomrule
\end{tabular}
\end{table}

%\begin{table}[h!]
%  \centering
%  \caption{The eigenvalue bias with transport-corrected MGXS.}
%  \label{tab:keff-bias-aniso} 
%  \begin{tabular}{c S[table-format=6.1] S[table-format=6.1] S[table-format=6.1]}
%  \toprule
%  & \multicolumn{3}{c}{{\bf FSR Discretization}} \\
%  \midrule
%  \multicolumn{1}{c}{{\bf \# Groups}} &
%  {\bf 1$\times$} & {\bf 4$\times$} & {\bf 16$\times$} \\
%  \midrule
%1 & 53 & 75 & 72 \\
%2 & 37 & 1 & 4 \\
%4 & -58 & -92 & -109 \\
%8 & -74 & -145 & -170 \\
%16 & -67 & -154 & -183 \\
%25 & -124 & -221 & -245 \\
%40 & -130 & -238 & -265 \\
%70 & -131 & -281 & -274 \\
%  \bottomrule
%\end{tabular}
%\end{table}

\begin{table}[h!]
  \centering
  \caption{The eigenvalue bias with isotropic-in-lab scattering.}
  \label{tab:keff-bias-iso-in-lab} 
  \begin{tabular}{c S[table-format=6.1] S[table-format=6.1] S[table-format=6.1]}
  \toprule
  & \multicolumn{3}{c}{{\bf FSR Discretization}} \\
  \midrule
  \multicolumn{1}{c}{{\bf \# Groups}} &
  {\bf 1$\times$} & {\bf 4$\times$} & {\bf 16$\times$} \\
  \midrule
1 & 80 & 55 & 66 \\
2 & 141 & 29 & 34 \\
4 & 27 & -43 & -57 \\
8 & 26 & -85 & -102 \\
16 & 35 & -91 & -111 \\
25 & -31 & -158 & -182 \\
40 & -38 & -174 & -202 \\
70 & -39 & -182 & -211 \\
  \bottomrule
\end{tabular}
\end{table}


In particular, the eigenvalue bias varies by up to 350 pcm between energy group structures and nearly 200 pcm between FSR discretizations. The use of isotropic in lab scattering in OpenMC reduces the magnitude of the bias by up to 100 pcm for seventy energy groups, indicating that the majority of the bias is unrelated to approximation error due to OpenMOC's isotropic in lab scattering kernel. Most importantly, the eigenvalue bias grows in magnitude and turns negative for increasingly fine energy group structures.

These results go against the intuition that finer energy discretization will necessarily permit more accurate multi-group eigenvalue transport calculations. Rather, they illustrate that even when cross sections are collapsed using the ``true'' scalar flux from Monte Carlo, reaction rates are not necessarily preserved. As a result, a non-negligible eigenvalue bias emerges between continuous energy and multi-group transport calculations when increasingly fine energy group and spatial discretization meshes are used.

%  -with enough groups (e.g., ultra-fine), MG and MC eigenvalues should match exactly


%%%%%%%%%%%%%%%%%%%%%%%%%%%%%%%%%%%%%%%%%%%%%%%%%%%%%%%%%%%%%%%%%%%%%%%%%%%%%%%
\subsection{Multi-Group Flux Bias}
\label{sec:case2-flux-bias}

first paragraph: present figures
-Fig.~\ref{fig:rel-err-energy} compares MG and MC flux by energy group
  -innermost and outermost rings of the fuel pin, and average across all rings
-Fig.~\ref{fig:rel-err-space} compares MG and MC flux by spatial zone
  -for three energy group ``ranges'' across all radial rings in the fuel pin
-describe the three energy ranges
  -range A: group for the U-238 capture resonance at 6.67 eV
  -range B: groups spanning the three lowest-lying U-238 capture resonances
  -range C: all U-238 resonance energy groups
  
second paragraph: analyze figures
-take analysis from my thesis
-the errors are largest for range A which most tightly encompasses a single resonance
-errors vary widely across fuel pin
-tie this into an argument of why this:
  (1) causes an eigenvalue bias
  (2) results from loss of angular information with scalar flux-weighted MGXS
-segue into next section on possible solutions

\begin{figure}[h!]
\centering
\includegraphics[width=\linewidth]{figures/rel-err-inner-outer}
\caption{The energy-dependent relative error of the OpenMOC scalar flux with respect to the reference OpenMC flux for the innermost, outermost and all FSRs.}
\label{fig:rel-err-energy}
\end{figure}

\begin{figure}[H]
\centering
\includegraphics[width=0.8\linewidth]{figures/rel-err-fuel-fsrs}
\caption{The spatially-varying relative error of the OpenMOC scalar flux with respect to the reference OpenMC flux in energy Ranges A, B, and C.}
\label{fig:rel-err-space}
\end{figure}