%%%%%%%%%%%%%%%%%%%%%%%%%%%%%%%%%%%%%%%%%%%%%%%%%%%%%%%%%%%%%%%%%%%%%%%%%%%%%%%
\section{Test Cases}
\label{sec:test-cases}
%%%%%%%%%%%%%%%%%%%%%%%%%%%%%%%%%%%%%%%%%%%%%%%%%%%%%%%%%%%%%%%%%%%%%%%%%%%%%%%

-remember to hammer home that all of this is due to heterogeneous effects


%%%%%%%%%%%%%%%%%%%%%%%%%%%%%%%%%%%%%%%%%%%%%%%%%%%%%%%%%%%%%%%%%%%%%%%%%%%%%%%
\subsection{Case 1: Resonance with Slowing Down Source}
\label{subsec:test-case1}

-cite nate's thesis~\cite{gibson2016thesis}
-this test case shows that reaction rates are not preserved in resonance groups
  -even when the MGXS is collapsed with the ``true'' reference scalar flux

%%%%%%%%%%%%%%%%%%%%%%%%%%%%%%%%%%%%%%%%%%%%%%%%%%%%%%%%%%%%%%%%%%%%%%%%%%%%%%%
\subsubsection{Benchmark Problem}
\label{subsubsec:benchmark-case1}

-this simple test case models a two-region PWR fuel pin cell problem
-energies around a single resonance of different widths (?) and heights

%%%%%%%%%%%%%%%%%%%%%%%%%%%%%%%%%%%%%%%%%%%%%%%%%%%%%%%%%%%%%%%%%%%%%%%%%%%%%%%
\subsubsection{Simulation Tools}
\label{subsubsec:sim-tools-case1}

-toy MOC/CPM (?) code used to generate an ultra-fine reference flux
-the reference ultra-fine flux is used to compute (1) reference single-group reaction rate and (2) a single-group MGXS for the resonance
-the single-group MGXS is re-inserted into the MOC/CPM (?) code for a single-group transport calculation
-the resulting single-group reaction rate is compared to the reference



%%%%%%%%%%%%%%%%%%%%%%%%%%%%%%%%%%%%%%%%%%%%%%%%%%%%%%%%%%%%%%%%%%%%%%%%%%%%%%%
\subsection{Case 2: A Critical PWR Fuel Pin}
\label{subsec:test-case2}

-cite my thesis~\cite{boyd2016thesis}


%%%%%%%%%%%%%%%%%%%%%%%%%%%%%%%%%%%%%%%%%%%%%%%%%%%%%%%%%%%%%%%%%%%%%%%%%%%%%%%
\subsubsection{Benchmark Problem}
\label{subsubsec:benchmark-case2}

first paragraph: a PWR fuel pin
-cite BEAVRS~\cite{horelik2013beavrs}
-fully-detailed four-region PWR fuel pin as shown in Fig.~\ref{fig:pin-materials}
  -1.6\% enriched UO$_2$ fuel, helium gap, zircaloy glad and light water moderator
-geometry was discretized into radial rings and sectors in Fig.~\ref{fig:pin-fsrs}
  -for both MGXS generation and multi-group eigenvalue calculations

-need to update figure with 16 rings in the fuel
  -match the plot of flux error by fuel FSR

\begin{table}[H]
  \centering
  \caption{2D fuel pin dimensions.}
  \label{table:pin-dimensions} 
  \begin{tabular}{l c}
  \toprule
  \multicolumn{1}{c}{\bf Material} &
  {\bf Dimension [cm]} \\
  \midrule
  Fuel Outer Radius & 0.39218 \\
  Gap Outer Radius &  0.40005 \\
  Clad Outer Radius & 0.45720 \\
  Pin Pitch &         1.25984 \\
  \bottomrule
\end{tabular}
\end{table}

\begin{figure}[h!]
\centering
\begin{subfigure}{.25\textwidth}
  \includegraphics[width=0.9\linewidth]{figures/pin-cell-simple}
  \caption{}
  \label{fig:pin-materials}
\end{subfigure}%
\begin{subfigure}{.25\textwidth}
  \centering
  \includegraphics[width=0.9\linewidth]{figures/pin-cell-8x8}
  \caption{}
  \label{fig:pin-fsrs}
\end{subfigure}
\caption{A PWR fuel pin cell materials (a) and FSRs (b).}
\label{fig:pin-cell}
\end{figure}


%%%%%%%%%%%%%%%%%%%%%%%%%%%%%%%%%%%%%%%%%%%%%%%%%%%%%%%%%%%%%%%%%%%%%%%%%%%%%%%
\subsection{Simulation Tools}
\label{subsubsec:sim-tools-case2}

first paragraph: OpenMC
-cite~\cite{romano2013openmc}
-continuous energy Monte Carlo neutron transport code
-\texttt{openmc.mgxs} to generate MGXS in 1 -- 70 energy groups
-generate reference eigenvalue, and scalar fluxes in each energy group structure

second paragraph: OpenMOC
-cite~\cite{boyd2014openmoc}
-multi-group method of characteristics (MOC) transport code
-2D heterogeneous calculations
-uses the MGXS generated by OpenMC for the exact same benchmark geometry
  -for various energy group structures (1 -- 70 energy groups)
-compare eigenvalues and multi-group fluxes from each calculation with reference solution