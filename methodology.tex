%%%%%%%%%%%%%%%%%%%%%%%%%%%%%%%%%%%%%%%%%%%%%%%%%%%%%%%%%%%%%%%%%%%%%%%%%%%%%%%
\section{Methodology}
\label{sec:methodology}
%%%%%%%%%%%%%%%%%%%%%%%%%%%%%%%%%%%%%%%%%%%%%%%%%%%%%%%%%%%%%%%%%%%%%%%%%%%%%%%

first paragraph: outline section
-section describes code methodology employed to understand impact of flux separability approx 
-hammer home that all of this is due to heterogeneous effects:
  -both test cases utilize a multi-region PWR fuel pin geometry
-a combination of analyses with ultra-fine deterministic and continuous energy Monte Carlo calculations to compute reference fluxes as well as MGXS for use and
-start with analysis in test case 1 (in~\autoref{subsec:test-case1}) of a fake resonance cross section in a 2-region PWR fuel pin to understand impact of approx. on reaction rate preservation within a single resonance energy group
-build up to test case 2 (in~\autoref{subsec:test-case2}) for a 4-region PWR fuel pin with multiple nuclides in to understand the global or ``compounding'' effect of the approx. across all energies in an eigenvalue calculation
-benchmarks and simulation codes are presented for each test case


%%%%%%%%%%%%%%%%%%%%%%%%%%%%%%%%%%%%%%%%%%%%%%%%%%%%%%%%%%%%%%%%%%%%%%%%%%%%%%%
\subsection{Test Case 1: Resonance with Slowing Down Source}
\label{subsec:test-case1}

first paragraph: outline section
-describe objective of this test case: to understand the impact of the flux separability 
-~\autoref{subsubsec:benchmark-case1} describes the benchmark problem
-~\autoref{subsubsec:sim-tools-case1} describes the simulation tools used
-cite nate's thesis~\cite{gibson2016thesis}
-this test case shows that reaction rates are not preserved in resonance groups
  -even when the MGXS is collapsed with the ``true'' reference scalar flux

%%%%%%%%%%%%%%%%%%%%%%%%%%%%%%%%%%%%%%%%%%%%%%%%%%%%%%%%%%%%%%%%%%%%%%%%%%%%%%%
\subsubsection{Benchmark Problem}
\label{subsubsec:benchmark-case1}

The first test case modeled a simple benchmark problem in which the reference flux could be computed precisely to demonstrate the existence of errors from approximately collapsing Eqn.~\ref{eqn:transport-eqn-ce} into Eqn.~\ref{eqn:transport-eqn-mg-separate} with the flux separability approximation. The first test case problem consisted of a unit cell of an infinite array of unclad fuel pins. The fuel material contained U-238 with an atom density of 0.022 $\nicefrac{\text{a}}{\text{b-cm}}$ and a purely scattering nuclide with a constant cross section of 0.176 cm\textsuperscript{-1}, an analog to oxygen in UO\textsubscript{2}. The moderator was a pure scatterer with a constant cross section of 1.23 cm\textsuperscript{-1}. The pin radius is 0.4 cm and the pitch was 1.26 cm. The source was given by the scatter source from the narrow resonance approximation:

\begin{dmath}
\label{eqn:test-source-ce}
Q(\mathbf{r},\mathbf{\Omega},E) = \frac{1}{4\pi} \frac{\Sigma_{p}(\mathbf{r})}{E}
\end{dmath}

% do you (instead) have a case where you analyze a single resonance with different peak/widths???

%%%%%%%%%%%%%%%%%%%%%%%%%%%%%%%%%%%%%%%%%%%%%%%%%%%%%%%%%%%%%%%%%%%%%%%%%%%%%%%
\subsubsection{Simulation Tools}
\label{subsubsec:sim-tools-case1}

A reference continuous energy flux was computed for the first test case by solving Eqn.~\ref{eqn:transport-eqn-mg-separate} for each energy point {\color{red}[using a toy MOC code? Spatial and angular discretization?]}. The reference reaction rates in each region were obtained by integrating the flux multiplied by a cross section over an energy range of interest and over the volume of the fuel pin. The total cross section was collapsed using the reference flux according to Eqn.~\ref{eqn:sigt-mg-scalar}, and the source was collapsed as follows:

\begin{dmath}
\label{eqn:test1-source-mg}
Q_{g}(\mathbf{r},\mathbf{\Omega}) = \int\displaylimits_{E_{g}}^{E_{g-1}} Q(\mathbf{r},\mathbf{\Omega},E)\mathrm{d}E
\end{dmath}

Eqn.~\ref{eqn:transport-eqn-mg-separate} was solved using {\color{red}[using a toy MOC code? Spatial and angular discretization?]} the collapsed total cross section and source. The collapsed reaction rate was obtained by volume-integrating the multi-group flux multiplied by the multi-group cross section over the fuel pin. Finally, the reaction rates obtained from the multi-group and reference calculations were compared in each energy group to identify any bias due to the flux separability approximation.


%%%%%%%%%%%%%%%%%%%%%%%%%%%%%%%%%%%%%%%%%%%%%%%%%%%%%%%%%%%%%%%%%%%%%%%%%%%%%%%
\subsection{Test Case 2: A Critical PWR Fuel Pin}
\label{subsec:test-case2}

first paragraph: outline section
-describe objective of this test case: to understand impact on flux separability approximation on eigenvalue calculations of LWRs
-~\autoref{subsubsec:benchmark-case2} describes the benchmark problem
-~\autoref{subsubsec:sim-tools-case2} describes the simulations tools used
-cite my thesis~\cite{boyd2016thesis}
-this test case shows that the flux separability approximation induces a biased eigenvalue
  -magnitude of bias depends on the number of energy groups used
  -reaction rates are not preserved in resonance groups
  -even when the MGXS is collapsed with the ``true'' reference scalar flux

%%%%%%%%%%%%%%%%%%%%%%%%%%%%%%%%%%%%%%%%%%%%%%%%%%%%%%%%%%%%%%%%%%%%%%%%%%%%%%%
\subsubsection{Benchmark Problem}
\label{subsubsec:benchmark-case2}

The second test case modeled a four-region PWR fuel pin, including  2.4\% UO\textsubscript{2} fuel, helium gap, zircaloy clad and borated light water moderator. This benchmark was derived from the BEAVRS PWR model and the material specifications are detailed by~\cite{horelik2013beavrs}. The geometric specifications are reproduced in~\autoref{table:pin-dimensions}. The benchmark geometry is illustrated in~\autoref{fig:pin-materials}. The fuel and moderator were each discretized into sixteen radial zones for spatial homogenization\footnote{Equal volume radial rings were used in the fuel; equally spaced radial rings were used in the moderator.} as shown in~\autoref{fig:pin-fsrs} for which unique MGXS were generated and employed in multi-group transport calculations.

\begin{table}[h!]
  \centering
  \caption{2D fuel pin dimensions.}
  \label{table:pin-dimensions} 
  \begin{tabular}{l c}
  \toprule
  \multicolumn{1}{c}{\bf Material} &
  {\bf Dimension [cm]} \\
  \midrule
  Fuel Outer Radius & 0.39218 \\
  Gap Outer Radius &  0.40005 \\
  Clad Outer Radius & 0.45720 \\
  Pin Pitch &         1.25984 \\
  \bottomrule
\end{tabular}
\end{table}

\begin{figure}[h!]
\centering
\begin{subfigure}{.25\textwidth}
  \includegraphics[width=0.9\linewidth]{figures/pin-cell-simple}
  \caption{}
  \label{fig:pin-materials}
\end{subfigure}%
\begin{subfigure}{.25\textwidth}
  \centering
  \includegraphics[width=0.9\linewidth]{figures/pin-cell-16x1}
  \caption{}
  \label{fig:pin-fsrs}
\end{subfigure}
\caption{The PWR fuel pin cell materials (a) and MGXS spatial homogenization zones (b) for the second test case benchmark.}
\label{fig:pin-cell}
\end{figure}

%%%%%%%%%%%%%%%%%%%%%%%%%%%%%%%%%%%%%%%%%%%%%%%%%%%%%%%%%%%%%%%%%%%%%%%%%%%%%%%
\subsection{Simulation Tools}
\label{subsubsec:sim-tools-case2}

The OpenMC continuous energy Monte Carlo code~\citep{romano2013openmc} was employed to generate multi-group cross sections, and reference multi-group reaction rates and fluxes, for the second test case. The \texttt{openmc.mgxs} Python module was used to tally multi-group cross sections in CASMO's seventy energy group structure~\citep{rhodes2006casmo} for each of the radial spatial zones illustrated in Fig.~\ref{fig:pin-fsrs} from a single eigenvalue calculation. The MGXS were tallied using the scalar flux (Eqn.~\ref{eqn:sigt-mg-scalar}) according to the flux separability approximation. The same Monte Carlo simulation was used to compute a reference eigenvalue, as well as reference multi-group fluxes and reaction rates for each energy group and spatial region. A total of 10 active and 100 inactive batches of 10\textsuperscript{6} particles per batch were simulated. 

It should be noted that two different OpenMC simulations were performed with different scattering kernels. The first simulation employed all of the normal scattering physics modeled in OpenMC. The second simulation used OpenMC's ``iso-in-lab'' feature\footnote{The ``iso-in-lab'' feature samples the outgoing neutron energy from the scattering laws prescribed by the continuous energy cross section library, but the outgoing neutron direction of motion is sampled from an isotropic in lab distribution.} to enforce isotropic in lab scattering. Although isotropic scattering is generally not a valid approximation for nuclear reactors, the ``iso-in-lab'' feature enabled comparisons between the reference eigenvalues and reaction rates produced by OpenMC and those computed from isotropic multi-group calculations with OpenMOC. Furthmore, the separate calculations enabled a comparison of the approximation errors due to isotropic in lab scattering and the flux separability approximation.

%The purpose of using a second, separate ``iso-in-lab'' simulation was twofold.

second paragraph: OpenMOC
-cite~\cite{boyd2014openmoc}
-multi-group method of characteristics (MOC) transport code
-2D heterogeneous calculations
-uses the MGXS generated by OpenMC for the exact same benchmark geometry
  -for various energy group structures (1 -- 70 energy groups)
-compare eigenvalues and multi-group fluxes from each calculation with reference solution