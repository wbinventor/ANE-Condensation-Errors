%%%%%%%%%%%%%%%%%%%%%%%%%%%%%%%%%%%%%%%%%%%%%%%%%%%%%%%%%%%%%%%%%%%%%%%%%%%%%%%
\section{Methodology}
\label{sec:methodology}
%%%%%%%%%%%%%%%%%%%%%%%%%%%%%%%%%%%%%%%%%%%%%%%%%%%%%%%%%%%%%%%%%%%%%%%%%%%%%%%

first paragraph: outline section
-section describes code methodology employed to understand impact of flux separability approx 
-hammer home that all of this is due to heterogeneous effects:
  -both test cases utilize a multi-region PWR fuel pin geometry
-a combination of analyses with ultra-fine deterministic and continuous energy Monte Carlo calculations to compute reference fluxes as well as MGXS for use and
-start with analysis in test case 1 (in~\autoref{subsec:test-case1}) of a fake resonance cross section in a 2-region PWR fuel pin to understand impact of approx. on reaction rate preservation within a single resonance energy group
-build up to test case 2 (in~\autoref{subsec:test-case2}) for a 4-region PWR fuel pin with multiple nuclides in to understand the global or ``compounding'' effect of the approx. across all energies in an eigenvalue calculation
-benchmarks and simulation codes are presented for each test case


%%%%%%%%%%%%%%%%%%%%%%%%%%%%%%%%%%%%%%%%%%%%%%%%%%%%%%%%%%%%%%%%%%%%%%%%%%%%%%%
\subsection{Test Case 1: Resonance with Slowing Down Source}
\label{subsec:test-case1}

first paragraph: outline section
-describe objective of this test case: to understand the impact of the flux separability 
-~\autoref{subsubsec:benchmark-case1} describes the be
-~\autoref{subsubsec:sim-tools-case1}
-cite nate's thesis~\cite{gibson2016thesis}
-this test case shows that reaction rates are not preserved in resonance groups
  -even when the MGXS is collapsed with the ``true'' reference scalar flux

%%%%%%%%%%%%%%%%%%%%%%%%%%%%%%%%%%%%%%%%%%%%%%%%%%%%%%%%%%%%%%%%%%%%%%%%%%%%%%%
\subsubsection{Benchmark Problem}
\label{subsubsec:benchmark-case1}

The first test case modeled a very simple problem in which the reference flux could be computed precisely to demonstrate the existence of errors from approximately collapsing Eqn.~\ref{eqn:transport-eqn-ce} into Eqn.~\ref{eqn:transport-eqn-mg-separate} with the flux separability approximation. The first test case problem consisted of a unit cell of an infinite array of unclad fuel pins. The fuel material contained U-238 with an atom density of 0.022 $\nicefrac{\text{a}}{\text{b-cm}}$ and a purely scattering nuclide with a constant cross section of 0.176 cm\textsuperscript{-1}, an analog to oxygen in UO\textsubscript{2}. The moderator was a pure scatterer with a constant cross section of 1.23 cm\textsuperscript{-1}. The pin radius is 0.4 cm and the pitch was 1.26 cm. The source was given by the scatter source from the narrow resonance approximation:

\begin{dmath}
\label{eqn:test-source-ce}
Q(\mathbf{r},\mathbf{\Omega},E) = \frac{1}{4\pi} \frac{\Sigma_{p}(\mathbf{r})}{E}
\end{dmath}

% do you (instead) have a case where you analyze a single resonance with different peak/widths???

%%%%%%%%%%%%%%%%%%%%%%%%%%%%%%%%%%%%%%%%%%%%%%%%%%%%%%%%%%%%%%%%%%%%%%%%%%%%%%%
\subsubsection{Simulation Tools}
\label{subsubsec:sim-tools-case1}

A reference continuous energy flux was computed for the first test case by solving Eqn.~\ref{eqn:transport-eqn-mg-separate} for each energy point {\color{red}[using a toy MOC code? Spatial and angular discretization?]}. The reference reaction rates in each region were obtained by integrating the flux multiplied by a cross section over an energy range of interest and over the volume of the fuel pin. The total cross section was collapsed using the reference flux according to Eqn.~\ref{eqn:sigt-mg-scalar}, and the source was collapsed as follows:

\begin{dmath}
\label{eqn:test1-source-mg}
Q_{g}(\mathbf{r},\mathbf{\Omega}) = \int\displaylimits_{E_{g}}^{E_{g-1}} Q(\mathbf{r},\mathbf{\Omega},E)\mathrm{d}E
\end{dmath}

Eqn.~\ref{eqn:transport-eqn-mg-separate} was solved using {\color{red}[using a toy MOC code? Spatial and angular discretization?]} the collapsed total cross section and source. The collapsed reaction rate was obtained by volume-integrating the multi-group flux multiplied by the multi-group cross section over the fuel pin. Finally, the reaction rates obtained from the multi-group and reference calculations were compared in each energy group to identify any bias due to the flux separability approximation.

%-toy MOC/CPM (?) code used to generate an ultra-fine reference flux
%-the reference ultra-fine flux is used to compute (1) reference single-group reaction rate and (2) a single-group MGXS for the resonance

%%%%%%%%%%%%%%%%%%%%%%%%%%%%%%%%%%%%%%%%%%%%%%%%%%%%%%%%%%%%%%%%%%%%%%%%%%%%%%%
\subsection{Test Case 2: A Critical PWR Fuel Pin}
\label{subsec:test-case2}

-cite my thesis~\cite{boyd2016thesis}


%%%%%%%%%%%%%%%%%%%%%%%%%%%%%%%%%%%%%%%%%%%%%%%%%%%%%%%%%%%%%%%%%%%%%%%%%%%%%%%
\subsubsection{Benchmark Problem}
\label{subsubsec:benchmark-case2}

first paragraph: a PWR fuel pin
-cite BEAVRS~\cite{horelik2013beavrs}
-fully-detailed four-region PWR fuel pin as shown in Fig.~\ref{fig:pin-materials}
  -1.6\% enriched UO$_2$ fuel, helium gap, zircaloy glad and light water moderator
-geometry was discretized into radial rings and sectors in Fig.~\ref{fig:pin-fsrs}
  -for both MGXS generation and multi-group eigenvalue calculations

-need to update figure with 16 rings in the fuel
  -match the plot of flux error by fuel FSR

\begin{table}[H]
  \centering
  \caption{2D fuel pin dimensions.}
  \label{table:pin-dimensions} 
  \begin{tabular}{l c}
  \toprule
  \multicolumn{1}{c}{\bf Material} &
  {\bf Dimension [cm]} \\
  \midrule
  Fuel Outer Radius & 0.39218 \\
  Gap Outer Radius &  0.40005 \\
  Clad Outer Radius & 0.45720 \\
  Pin Pitch &         1.25984 \\
  \bottomrule
\end{tabular}
\end{table}

\begin{figure}[h!]
\centering
\begin{subfigure}{.25\textwidth}
  \includegraphics[width=0.9\linewidth]{figures/pin-cell-simple}
  \caption{}
  \label{fig:pin-materials}
\end{subfigure}%
\begin{subfigure}{.25\textwidth}
  \centering
  \includegraphics[width=0.9\linewidth]{figures/pin-cell-16x8}
  \caption{}
  \label{fig:pin-fsrs}
\end{subfigure}
\caption{A PWR fuel pin cell materials (a) and FSRs (b).}
\label{fig:pin-cell}
\end{figure}


%%%%%%%%%%%%%%%%%%%%%%%%%%%%%%%%%%%%%%%%%%%%%%%%%%%%%%%%%%%%%%%%%%%%%%%%%%%%%%%
\subsection{Simulation Tools}
\label{subsubsec:sim-tools-case2}

first paragraph: OpenMC
-cite~\cite{romano2013openmc}
-continuous energy Monte Carlo neutron transport code
-\texttt{openmc.mgxs} to generate MGXS in 1 -- 70 energy groups
-generate reference eigenvalue, and scalar fluxes in each energy group structure
-mention that iso-in-lab scattering can be used in OpenMC
  -permits the elimination of the need for an anisotropic scattering kernel in downstream multi-group transport code, or a transport correction, to capture effects from anisotropic scattering
  -also permits quantification of anisotropic scattering effects
  -will be used in later sections

second paragraph: OpenMOC
-cite~\cite{boyd2014openmoc}
-multi-group method of characteristics (MOC) transport code
-2D heterogeneous calculations
-uses the MGXS generated by OpenMC for the exact same benchmark geometry
  -for various energy group structures (1 -- 70 energy groups)
-compare eigenvalues and multi-group fluxes from each calculation with reference solution