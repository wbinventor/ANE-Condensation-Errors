%%%%%%%%%%%%%%%%%%%%%%%%%%%%%%%%%%%%%%%%%%%%%%%%%%%%%%%%%%%%%%%%%%%%%%%%%%%%%%%
\section{Introduction}
\label{sec:intro}
%%%%%%%%%%%%%%%%%%%%%%%%%%%%%%%%%%%%%%%%%%%%%%%%%%%%%%%%%%%%%%%%%%%%%%%%%%%%%%%

\begin{itemize}
\item Derivation of multigroup transport equation
\item Approximation of scalar-flux weighting the total cross section
\item Higher fidelity scalar flux approximations commonplace
\item This paper will show consequences of that assumption
\end{itemize}

-somewhwere discuss how 

first paragraph: motivation
-motivate need for high-fidelity transport methods
-compare contrast MC with deterministic methods
-adapt from thesis intro

second paragraph: MGXS
-need MGXS for deterministic methods
-need the flux to compute MGXS

third paragraph: MGXS with MC
-use MC for reference calculation
-use MC to compute MGXS with a statistical proxy to the flux
-still approx. even with ``true'' flux-weighting

fourth paragraph: outline paper
-segue that this paper investigates the impact of one such approx.
  -namely, the impact of the flux separability approximation
-Sec.~\ref{sec:flux-separability} introduces the flux separability approximation
-Sec.~\ref{sec:methodology}
-Sec.~\ref{sec:preservation}
-Sec.~\ref{sec:bias} introduces the impact of the eigenvalue bias
-Sec.~\ref{sec:solutions}
-Sec.~\ref{sec:conclusions}