%%%%%%%%%%%%%%%%%%%%%%%%%%%%%%%%%%%%%%%%%%%%%%%%%%%%%%%%%%%%%%%%%%%%%%%%%%%%%%%
\section{Introduction}
\label{sec:intro}
%%%%%%%%%%%%%%%%%%%%%%%%%%%%%%%%%%%%%%%%%%%%%%%%%%%%%%%%%%%%%%%%%%%%%%%%%%%%%%%

The nuclear reactor physics community has long strived for deterministic neutron transport-based tools for whole-core reactor analysis. A key challenge for whole-core multi-group transport methods is accurate reactor agnostic multi-group cross section (MGXS) generation. The MGXS generation process applies a series of approximations to produce spatially homogenized and energy condensed MGXS in each spatial zone and energy group. Many approximations related to multi-group theory, including the selection of discretized energy group structures and the truncation of the Legendre expansion of the multi-group scattering kernel, are widely studied in the literature. However, the practical impact of the flux separability approximation, which permits the use of the scalar rather than the angular neutron flux to weight the continuous energy cross sections, is less understood. This paper investigates the flux separability approximation and quantifies its significance for heterogeneous PWR problems.

In the generation of MGXS, increasingly higher-fidelity methods are becoming widespread.  Most significantly, this includes Monte Carlo and ultra-fine deterministic methods, which provide a high-quality continuous-energy (or near-continuous-energy) spectrum for some reference problem.  Monte Carlo methods have increasingly been used to generate few group constants for coarse mesh diffusion, most notably by the Serpent MC code \citep{serpent2013manual}, and to a much lesser extent, for high-fidelity neutron transport methods \citep{redmond1997multigroup, nelson2014improved, cai2014condensation, boyd2016thesis}.

The need to know the flux accurately to generate MGXS is the primary difficulty in generating appropriate nuclear data.  However, as this work explores, the use of higher-fidelity collapsing spectra exposes other errors which are derived from approximations made in the formulation of the multi-group equations. This paper shows that even when the ``true'' scalar flux is used to generate MGXS, the flux separability approximation results in a non-negligible eigenvalue bias between continuous energy and multi-group calculations due to over-prediction of U-238 capture in resonance energy groups.

The content in this paper is organized as follows. The flux separability approximation is introduced in~\autoref{sec:flux-separability}.  Two benchmark problems---one using simplified ultra-fine methods and a more realistic one using Monte Carlo methods---are presented and explored in~\autoref{sec:test-case1} and~\autoref{sec:test-case2}, rigorously quantifying the impact of the approximation on PWR problems.  The former examines demonstrates that scalar-flux-weighting of cross sections cannot preserve the reaction rate in a single resonance group.  The latter investigates the flux separability approximation in the context of a fully-detailed, critical PWR fuel pin cell.  Angularly-dependent MGXS and SuPerHomog\'{e}n\'{e}isation (SPH) factors are explored in~\autoref{sec:solutions} as two possible corrections to this approximation, but both methods are also noted to have significant shortcomings.
