%%%%%%%%%%%%%%%%%%%%%%%%%%%%%%%%%%%%%%%%%%%%%%%%%%%%%%%%%%%%%%%%%%%%%%%%%%%%%%%
\section{Potential Solutions}
\label{sec:solutions}
%%%%%%%%%%%%%%%%%%%%%%%%%%%%%%%%%%%%%%%%%%%%%%%%%%%%%%%%%%%%%%%%%%%%%%%%%%%%%%%

The preceding sections showed that the flux separability approximation leads to reaction rate errors in energy groups with large U-238 capture resonances, which results in a non-negligible bias in the eigenvalue. This section reviews two approaches to counter the impact of the flux separability approximation. \autoref{subsec:angular-mgxs} and \autoref{subsec:alternate-angle} investigate the use of angularly-dependent MGXS in the context of the first test case. \autoref{subsec:sph} introduces SuPerHomog\'{e}n\'{e}isation factors to systematically adjust the scalar flux-weighted MGXS to preserve reaction rates in the context of the second test case.

%The preceding results demonstrated that using the ``true'' scalar flux spectrum from an ultra-fine deterministic or Monte Carlo simulation to perform energy condensation and spatial homogenization will not necessarily preserve group reaction rates.

%%%%%%%%%%%%%%%%%%%%%%%%%%%%%%%%%%%%%%%%%%%%%%%%%%%%%%%%%%%%%%%%%%%%%%%%%%%%%%%
\subsection{Angularly-Dependent Total MGXS}
\label{subsec:angular-mgxs}

The flux separability approximation introduced in~\autoref{sec:flux-separability} led to the use of the scalar rather than the angular flux to condense the total cross section in energy. The mathematically proper treatment would instead use the angular flux to condense the total MGXS in angle, energy and space, resulting in angularly-dependent total MGXS. This section shows that using fully angularly-dependent data causes the bias in the group reaction rates to vanish for the first test case benchmark.

As was noted in~\autoref{sec:flux-separability}, the most important aspect of the physics to account for in an LWR fuel pin is whether neutrons are entering the region through the fuel pin or directly from the moderator (see~\autoref{fig:incoming-outgoing}). This self-shielding effect cannot be properly modeled by the simple application of angularly-dependent to a single region fuel pin. In this case, a cross section for a given angle would be applied to both neutrons entering the fuel pin from the moderator and to neutrons exiting the fuel pin on the opposite side. Thus, the fuel pin must also be spatially discretized in order to resolve the error.

The impact of angularly-dependent MGXS was investigated for first test case benchmark of a simple two-region pin cell. The reference solution was computed on an ultra-fine energy mesh and was used to collapse cross sections. Cross sections for the left hand side of the transport equation were collapsed separately for each angle, weighted by the angular flux for that angle. The geometry was discretized with equally-spaced azimuthal sectors, equal volume fuel rings, and equal width moderator rings.  A depiction of this discretization with six azimuthal sectors, three fuel rings and two moderator rings is shown in~\autoref{fig:pin-disc}.

\begin{figure}
\centering
\caption{Depiction of pin discretization with 6 azimuthal sectors, 3 fuel rings, and 2 moderator rings.}
\label{fig:pin-disc}
\includegraphics[width=0.6\linewidth]{figures/pin-discrete}
\end{figure}

The reaction rate errors in resonance groups with angular flux-weighted MGXS are presented in~\autoref{tab:angular-mgxs-1} and \autoref{tab:angular-mgxs-2}.  The former does not discretize the pin, while the latter uses 5 fuel rings and 8 sectors.  In both cases, three moderator rings are used.  Regardless of pin discretization, weighting with the scalar flux leads to errors over 1\% for the groups containing the low-lying resonances of U-238.  With angular flux weighting, the error is reduced by about an order of magnitude when the pin is discretized; however, without pin discretization, the errors are not reduced.  To see the trend of pin discretization compared to error, see~\autoref{fig:angular-mgxs-error}.

\begin{table}[h!]
  \centering
  \caption{U-238 resonance range reaction rate percent relative errors computed with scalar and angular flux-weighted MGXS for the first test case benchmark without pin discretization. \red{Add data!}}
  \label{tab:angular-mgxs-1}
  \begin{tabular}{c c c c c}
  \toprule
  & & \multicolumn{2}{c}{\textbf{Error [\%]}} \\
  \cline{3-4}
  \textbf{Group} & \textbf{\boldmath$E_{max}$ [eV]} & \textbf{Scalar} & \textbf{Angular} \\
  \midrule
  15 & 9118.00 & 1.05E-01 & 1.83E-01 \\
  16 & 5530.00 & 1.42E-01 & 2.28E-01 \\
  17 & 3519.10 & 3.29E-01 & 4.85E-01 \\
  18 & 2239.45 & 4.57E-01 & 6.42E-01 \\
  19 & 1425.10 & 4.61E-01 & 6.34E-01 \\
  20 & 906.899 & 2.94E-01 & 4.08E-01 \\
  21 & 367.263 & 5.43E-01 & 7.12E-01 \\
  22 & 148.729 & 5.98E-01 & 7.79E-01 \\
  23 & 75.5014 & 4.87E-01 & 6.21E-01 \\
  24 & 48.0520 & 9.64E-01 & 1.25E+00 \\
  25 & 27.7000 & 9.84E-01 & 1.26E+00 \\
  26 & 15.9680 & 1.19E-03 & 2.77E-03 \\
  27 & 9.87700 & 1.04E+00 & 1.31E+00 \\
  \bottomrule
\end{tabular}
\end{table}

\begin{table}[h!]
  \centering
  \caption{U-238 resonance range reaction rate percent relative errors computed with scalar and angular flux-weighted MGXS for the first test case benchmark with 5 fuel rings and 8 azimuthal sectors.}
  \label{tab:angular-mgxs-2}
  \begin{tabular}{c c c c c}
  \toprule
  & & \multicolumn{2}{c}{\textbf{Error [\%]}} \\
  \cline{3-4}
  \textbf{Group} & \textbf{\boldmath$E_{max}$ [eV]} & \textbf{Scalar} & \textbf{Angular} \\
  \midrule
  15 & 9118.00 & 1.29E-01 & 9.93E-03 \\
  16 & 5530.00 & 1.75E-01 & 1.23E-02 \\
  17 & 3519.10 & 4.04E-01 & 2.68E-02 \\
  18 & 2239.45 & 5.67E-01 & 3.70E-02 \\
  19 & 1425.10 & 5.75E-01 & 3.64E-02 \\
  20 & 906.899 & 3.65E-01 & 2.20E-02 \\
  21 & 367.263 & 6.99E-01 & 4.87E-02 \\
  22 & 148.729 & 7.75E-01 & 5.78E-02 \\
  23 & 75.5014 & 6.47E-01 & 4.90E-02 \\
  24 & 48.0520 & 1.24E+00 & 1.10E-02 \\
  25 & 27.7000 & 1.27E+00 & 1.18E-02 \\
  26 & 15.9680 & 1.29E-03 & 3.45E-02 \\
  27 & 9.87700 & 1.34E+00 & 1.21E-01 \\
  \bottomrule
\end{tabular}
\end{table}

\begin{figure}
\caption{Errors using angular flux weighted multi-group cross sections as a function of pin discretization.}
\label{fig:angular-mgxs-error}
\includegraphics[width=\linewidth]{figures/pin-disc-rel-err}
\end{figure}

This demonstration fully diagnoses the errors observed previously, but angular-dependent cross sections are not a realistic option for most applications.  As demonstrated here, a great number of spatial discretization is needed, which both increases the computational time of the transport solution and increases the memory footprint of the cross section data.  Also, angular-dependent cross sections themselves require a significant amount of additional memory to store, compared to scalar flux weighted cross sections.  Furthermore, the generation of angular-dependent cross sections requires knowledge of the angular flux at the fine-group or continuous-energy level, and this is not available in most workflows.  Thus, angular-dependent cross sections are likely not the ultimate solution to the problem highlighted in this paper.

%As a result, multi-group codes are unable to reproduce the correct angular dependence of the neutron flux with scalar flux-weighted MGXS. Furthermore, flux separability is an approximation which may lead to non-trivial errors in downstream multi-group calculations, such as the eigenvalue bias observed in~\autoref{sec:test-case2}.

%%%%%%%%%%%%%%%%%%%%%%%%%%%%%%%%%%%%%%%%%%%%%%%%%%%%%%%%%%%%%%%%%%%%%%%%%%%%%%%
\subsection{Alternate Approach to Angular-Dependent Cross Sections}
\label{subsec:alternate-angle}

Multi-group transport solvers currently available do not typically allow for the use of angular-dependent cross sections in the collision term of the left hand side of the transport equation.  However, one can move this dependence into the source terms on the right hand side.  By subtracting the collision term and adding a modified collision term to both sides of Eqn.~\ref{eqn:transport-eqn-mg}, an equivalent multi-group transport equation is
%
\begin{dmath}
\mathbf{\Omega} \cdot \nabla \psi_{g}(\mathbf{r},\mathbf{\Omega}) + \hat{\Sigma}_{t,g}(\mathbf{r})\psi_{g}(\mathbf{r},\mathbf{\Omega}) = Q_{g}(\mathbf{r},\mathbf{\Omega}) + \left(\hat{\Sigma}_{t,g}(\mathbf{r})-\Sigma_{t,g}(\mathbf{r},\mathbf{\Omega})\right)\psi_{g}(\mathbf{r},\mathbf{\Omega}) \quad .
\end{dmath}
%
The added source term can also be expressed in a truncated spherical harmonics expansion:
%
\begin{dmath}
\mathbf{\Omega} \cdot \nabla \psi_{g}(\mathbf{r},\mathbf{\Omega}) + \hat{\Sigma}_{t,g}(\mathbf{r})\psi_{g}(\mathbf{r},\mathbf{\Omega}) = Q_{g}(\mathbf{r},\mathbf{\Omega}) + 
\sum_{l=0}^L \sum_{m=-l}^{l} Y_l^m (\mathbf{\Omega})
\left(\hat{\Sigma}_{t,g}(\mathbf{r})-\Sigma_{t,g,lm}(\mathbf{r})\right)\psi_{g,lm}(\mathbf{r}) .
\end{dmath}
%
These forms of the transport equation can be solved by most multi-group transport solvers, provided that they allow an iterated angular-dependent or angular moment source.  In one dimension, when the spherical harmonic simplifies to a Legendre polynomial, the source term can be built into the Legendre-expanded scattering matrix, as shown in \citep{macfarlane2000njoy} \red{[and Bell-Glasstone]}. 

In this formulation, the choice of $\hat{\Sigma}_{t, g}$ is arbitrary.  Its choice is often influenced by the transport approximation, where it is desired to preserve scattering moments higher than what are included in the scattering source \citep{bell1967transport}.  The simplest option is known as consistent-$P$,
%
\begin{dmath}
\hat{\Sigma}_{t, g}(\mathbf{r}) = \Sigma_{t,g}(\mathbf{r}) \quad ,
\end{dmath}
%
where $\Sigma_{t,g}$ is the isotropic component of the total cross section and is equivalent to the definition in Eqn.~\ref{eqn:sigt-mg-scalar}. 

This approach has many of the same issues as using angular-dependent cross sections directly.  The data associated with a fully angular-dependent source is the same as with angular-dependent cross sections and is likely impractical for most applications.  Even the moments become cumbersome when more than a few spherical harmonics are included in the expansion.  Furthermore, the previous conclusion that spatial discretization is necessary to capture angular effects is not alleviated by moving the dependence to the right hand side.  The generation of the source terms also require knowledge of the detailed angular flux (or its moments) at the fine-group or continuous-energy level.

%%%%%%%%%%%%%%%%%%%%%%%%%%%%%%%%%%%%%%%%%%%%%%%%%%%%%%%%%%%%%%%%%%%%%%%%%%%%%%%
\subsection{SuPerHomog\'{e}n\'{e}isation Factors}
\label{subsec:sph}

SuPerHomog\'{e}n\'{e}isation (SPH) factors are an alternative method to reduce heterogeneous resonant reaction rate errors. SPH factors were first proposed by~\cite{hebert1993consistent} to preserve reaction rates during energy condensation and spatial homogenization. SPH factors have traditionally been applied to spatially-homogenized few-group MGXS for coarse mesh diffusion applications. This section introduces SPH factors as an equivalence method between continuous energy Monte Carlo and deterministic multi-group transport methods for the second test case benchmark.

%%%%%%%%%%%%%%%%%%%%%%%%%%%%%%%%%%%%%%%%%%%%%%%%%%%%%%%%%%%%%%%%%%%%%%%%%%%%%%%
\subsubsection{SPH Algorithm}
\label{subsubsec:sph-algorithm}

%SPH is a relatively simple-to-implement method which was evaluated in OpenMOC as discussed in the following sections.

The SPH algorithm enforces reaction rate preservation between a reference fine-mesh transport problem and a corresponding coarse mesh transport or diffusion problem in energy and space. As a result, the SPH factor algorithm requires knowledge of a reference source that is used in a multi-group fixed source solver to derive multiplicative factors that adjust the total MGXS to force neutron balance. In particular, the SPH scheme postulates the existence of a set of factors $\mu_{k,g}$ for each spatial zone $k$ and energy group $g$ which force the streaming and collision terms in the transport equation to balance with a fixed source $Q_{k,g}$:

\begin{dmath}
\label{eqn:sph-transport-eqn}
\mathbf{\Omega} \cdot \nabla \psi_{g}(\mathbf{r},\mathbf{\Omega}) + \mu_{k,g}\Sigma_{t,k,g}\psi_{g}(\mathbf{r},\mathbf{\Omega}) = Q_{k,g}(\mathbf{\Omega}) \quad .
\end{dmath}

The SPH factors are applied to correct the total MGXS in each region and group. The fixed source $Q_{k,g}$ is computed from the reference fine-mesh solution. In this case, the fixed source is treated as the sum of scattering and fission production sources in each energy group and spatial zone. Given the fixed source and total MGXS from MC,~\autoref{eqn:sph-transport-eqn} may be solved using any multi-group transport method, such as MOC.

%The challenge is to devise estimates to the true SPH factors $\mu_{k,g}$ which adequately preserve reaction rates. 

%For example, continuous energy Monte Carlo can be used to compute reference multi-group fluxes and MGXS, which are then combined to compute an isotropic source as follows:

%\begin{dmath}
%\label{eqn:sph-source}
%Q_{k,g}(\mathbf{\Omega}) = \frac{1}{4\pi} \sum_{g'=1}^{G} \Sigma_{s,k,g' \rightarrow g}\phi_{k,g'} + \frac{\chi_{k,g}}{4\pi k_{eff}}\sum_{g'=1}^{G} \nu\Sigma_{f,k,g'}\phi_{k,g'}
%\end{dmath}

An iterative algorithm is used to estimate SPH factors from a series of multi-group fixed source calculations. First, the estimates $\mu_{k,g}^{(n)}$ at iteration $n$ are introduced as a correction factor for the total cross section in~\autoref{eqn:sph-transport-eqn}:
%
\begin{dmath}
\label{eqn:sph-transport-eqn-iterate}
\mathbf{\Omega} \cdot \nabla \psi_{g}^{(n)}(\mathbf{r},\mathbf{\Omega}) + \mu_{k,g}^{(n-1)}\Sigma_{t,k,g}\psi_{g}^{(n)}(\mathbf{r},\mathbf{\Omega}) = Q_{k,g}(\mathbf{\Omega}) \quad .
\end{dmath}
%
A multi-group transport code is used to solve~\autoref{eqn:sph-transport-eqn-iterate} with angular and volume integration to compute the scalar flux distribution. The SPH factors $\mu_{k,g}^{(n)}$ are found from the ratio of the reference Monte Carlo scalar flux $\phi_{k,g}^{MC}$ to the flux $\phi_{k,g}^{(n)}$ computed from the fixed source calculation at iteration $n$,
%
\begin{dmath}
\label{eqn:sph-update}
\mu_{k,g}^{(n)} = \frac{\phi_{k,g}^{MC}}{\phi_{k,g}^{(n)}} \quad .
\end{dmath}
%
where the factors are initialized to unity on the first iteration.

The SPH factors are used to find a total MGXS which forces neutron balance in~\autoref{eqn:sph-transport-eqn-iterate}. The initial total MGXS $\Sigma_{t,k,g}^{(0)}$ is computed from the reference MC flux and total reaction rate tallies. The SPH factors are then used to obtain a corrected total MGXS $\Sigma_{t,k,g}^{(n)}$ on each iteration:
%
\begin{dmath}
\label{eqn:sph-update-sigt}
\Sigma_{t,k,g}^{(n)} = \mu_{k,g}^{(n-1)}\Sigma_{t,k,g}^{(0)} \quad .
\end{dmath}

The series of fixed source problems defined by~\autoref{eqn:sph-transport-eqn-iterate} are solved until the SPH factors converge. The scattering matrix and fission production cross section are used to compute the reference fixed source, but are not needed in the iterative scheme defined in Eqn.~\ref{eqn:sph-transport-eqn-iterate}. However, the converged SPH factors must be applied to the scattering matrix and fission production cross sections to produce a fully-corrected MGXS library for downstream eigenvalue calculations.

The SPH iteration algorithm described here is summarized in~\autoref{alg:sph}. As presently posed, there is no unique solution to the set of SPH factors which preserve reaction rates. A unique solution may be found by forcing the factors to be unity in non-fissile zones (\textit{e.g.}, moderator, clad and gap). This approach is motivated by the fact that resonances which lead to self-shielding errors -- such as the U-238 capture resonances -- are generally from isotopes in the fuel. However, the reaction rates in non-fissile zones will not be preserved since the MGXS in these zones remain uncorrected, but these errors are dominated by those in the fuel.

\begin{algorithm}[h]
\caption{SPH Factor Algorithm}
\label{alg:sph}
\begin{algorithmic}[1]
  \State Initialize MGXS from MC tallies
  \State Compute reference source from MC flux and MGXS
  \State Initialize SPH factors to unity
  \While{SPH factors are not converged}
    \State Update total MGXS with SPH factors
    \State Solve fixed source transport problem\footnotemark
    \State Compute new SPH factors
  \EndWhile
  \State Update all MGXS with SPH factors
\end{algorithmic}
\end{algorithm}

\footnotetext{A series of $G$ independent fixed source problems may be solved for each of the $G$ groups. Alternatively, a single fixed source problem may simultaneously solve for all $G$ groups, as is done in OpenMOC.}


%%%%%%%%%%%%%%%%%%%%%%%%%%%%%%%%%%%%%%%%%%%%%%%%%%%%%%%%%%%%%%%%%%%%%%%%%%%%%%%
\subsubsection{Results}
\label{subsubsec:sph-results}

The second test case benchmark was modeled with SPH-corrected MGXS for the FSRs in the fuel and compared to the results presented in~\autoref{sec:test-case2}. The MGXS were computed using isotropic in lab scattering and a spatial tally mesh corresponding to an FSR mesh with 16 radial rings in both fuel and moderator. The OpenMOC fixed source and eigenvalue calculations were performed with 128 azimuthal angles and 0.01 cm track spacing. The OpenMOC fixed source calculations were converged to 10$^{-5}$ on the average FSR scalar flux.

The eigenvalue bias $\Delta\rho$ between OpenMC and OpenMOC for SPH-corrected MGXS is presented in~\autoref{tab:keff-bias-sph}, which can be compared to the bias without SPH in~\autoref{tab:keff-bias-iso-in-lab}. The results illustrate a large reduction in the eigenvalue bias; in particular, the bias of -210 pcm was reduced to just -3 pcm for the finest energy and spatial discretization.

\begin{table}[h!]
  \centering
  \caption{The eigenvalue bias with SPH-corrected MGXS.}
  \label{tab:keff-bias-sph} 
  \begin{tabular}{c S[table-format=6.1] S[table-format=6.1] S[table-format=6.1]}
  \toprule
  & \multicolumn{3}{c}{{\bf FSR Discretization}} \\
  \cline{2-4}
  \multicolumn{1}{c}{{\bf \# Groups}} &
  {\bf 1$\times$} & {\bf 4$\times$} & {\bf 16$\times$} \\
  \midrule
1 & 19 & -18 & -14 \\
2 & 25 & -14 & -6 \\
4 & 7 & 2 & 1 \\
8 & 4 & -0 & 2 \\
16 & 5 & 0 & 4 \\
25 & 5 & 2 & -1 \\
40 & 4 & 3 & -2 \\
70 & 4 & 2 & -3 \\
  \bottomrule
\end{tabular}
\end{table}

Although these results illustrate much better agreement between OpenMC and OpenMOC, a non-negligible bias remains in most cases. It is postulated that the remaining bias may be due to the fact that the reaction rate balance enforced with SPH factors assumes that the eigenvalue calculation with a multi-group method will produce the same neutron source distribution as continuous energy Monte Carlo. However, the eigenvalue source is not necessarily conserved since approximation errors from spatial and angular discretization will impact the multi-group method's solution.

%In summary, the eigenvalues between continuous energy and multi-group transport calculations will identically match if the multi-group transport method computes the same eigenvalue source with SPH-corrected MGXS as that found by the continuous energy method.

The error of OpenMOC's 70-group flux with respect to the reference OpenMC flux is displayed in~\autoref{fig:rel-err-energy-sph}. The errors are shown for the innermost and outermost FSRs in the fuel, along with the average error across all FSRs in the fuel. The flux error is greatly reduced with SPH-corrected MGXS, and is nearly flat in energy. The improvement in OpenMOC's flux -- especially in those energy groups with large U-238 capture resonances -- is responsible for the reduction in the eigenvalue bias with SPH-corrected MGXS presented in~\autoref{tab:keff-bias-sph}.

\begin{figure}[h!]
\centering
\includegraphics[width=\linewidth]{figures/rel-err-inner-outer-sph}
\caption{The energy-dependent relative error of the OpenMOC scalar flux computed with SPH-corrected MGXS with respect to the reference OpenMC flux for the innermost, outermost and all FSRs.}
\label{fig:rel-err-energy-sph}
\end{figure}

Although SPH factors can greatly improve the agreement between OpenMC and OpenMOC, the SPH approach suffers from a number of shortcomings. First, the SPH approach requires knowledge of the true eigenvalue source distribution. If a reference source must first be computed with a continuous energy method in order to compute SPH-corrected MGXS, then there is no reason to perform a subsequent multi-group calculation since the solution is already known from the reference calculation. Second, the SPH scheme is wholly dependent on the spatial discretization used by downstream deterministic multi-group methods. In particular, the reference source must be calculated from the reference continuous energy method in each of the spatial mesh cells used by the SPH iteration scheme. Lastly, the SPH scheme attempts to preserve reaction rates irregardless of the types of approximation errors which may lead to bias between continuous energy and multi-group transport codes. The SPH scheme simultaneously ``corrects'' the MGXS to account for errors deriving from the spatial, angular and energy discretization and the treatment of the scattering kernel, in addition to those errors resulting from the flux separability approximation.

As a result of the shortcomings to the SPH approach, it is unclear whether the SPH factors may be broadly applied to correct for the flux separability approximation in MGXS generated from MC. Future work should investigate whether a universal set of SPH factors may be tabulated for known geometries (\textit{e.g.}, PWR fuel pins). For example, if the SPH factors in the resonance groups are relatively invariant to the fuel enrichment, moderator density, burnup  and neighboring pin types, then a single set of SPH factors may be computed for an infinite pin cell and applied to each fuel pin in heterogeneous PWR lattice and full-core calculations.

%n particular, the SPH scheme requires knowledge of the reference source distribution, is dependent on the spatial discretization mesh, and is indiscriminate between various sources of approximation error.
